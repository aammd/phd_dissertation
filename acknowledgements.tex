% There is currently a problem with spacing somewhere so that Table of
% Contents, List of Tables, and List of Figures have the wrong amount
% of space.  Others are OK though...
\chapter*{Acknowledgements}
\addcontentsline{toc}{chapter}{Acknowledgements}

\begin{quoteshrink}
% \begin{center}
  \emph{Se a ci\^encia quer ser a verdadeira,\\
que ci\^encia mais verdadeira que a das cousas sem ci\^encia?\\
Fecho os olhos e a terra dura sobre que me deito\\
Tem uma realidade t\~ao real que at\'e as minhas costas a sentem.\\ 
N\~ao preciso de racioc\'{\i}inio onde tenho esp\'aduas}\\
  \hfill --Fernando Pessoa
% \end{center}
\end{quoteshrink}


I thank first of all my supervisor, Dr. Diane Srivastava -- for years I have loved the topic of ecology, but it was by studying under Diane that I learned how to reason about it. I could not have completed this thesis without her encouragement, tenacity and insight into ecology.  I am also grateful to my supervisory committee -- Dr. Leticia Aviles, Dr. Mary O'Connor, and Dr. Dolph Schluter -- for their advice and helpful critiques throughout this process. 

My labmates have been a constant source of encouragement and support from the moment I joined the lab. I especially thank my labmate Dr. Robin Lecraw, who worked alongside me during fieldwork for Chapters 3 and 4. I thank Sarah Amundrud and Melissa Guzman for our many discussions, and for their feedback on my writing. I'm also very grateful to our lab's postdocs -- Drs. Jana Petermann, Pavel Kratina, and Angelica Gonzalez -- for their advice and inspiration. Most of all I thank my intellectual twin, Dr. Alathea Letaw, for many constructive conversations and her near-constant encouragement.

Esse trabalho seria impossivel sem meus colegos brasilerios. Agrade\c{c}o muito os contribu\c{c}\~oes, o apoio, e o amizade de tudo mundo. Agrade\c{c}o o professor Gustavo Romero (Unicamp), quem ajoudou com os experimentos de Cap\'itulo 2 e 3. Tambem agrade\c{c}o varios membros do seu labrotorio: Paula Omena, Tiago Bernab\'e, Gustavo Caue, Maraisa Braga e Aline Nishi. Eu tambem agrade\c{c}o e reconhe\c{c}o os moradores do Parque Estadual Ilha do Cardoso et Parque National Jurubatiba, pelo acolher-me em suas casas. Eu tambem tinha o privil\'egio de trabalhar com a Groupo de Limnologia (UFRJ), com professor Vinicius Farjalla. Agrade\c{c}o muito meus amigos e colegos l\'a: Nicolas Marino, Aliny Pires, Juliana Leal e Alice Campos. Sem eles o trabalho de Cap\'itulo 1 teria sido imposivel. Acima de tudo, agrade\c{c}o meus assistentes de campo: Aline Nishi (Cap\'itulo 2) e meu querido amigo e ajudante Pedro Trasmonte (Cap\'itulo 1 e 3). Obrigad\~ao para tudo!

I am also grateful to the wider UBC community for their support and inspiration throughout my PhD. Thank you to our whole intelligent, creative department. In particular, the biodiversity discussion group and its many attendees taught me how to think like an ecologist. The amount of practical assistance and guidance I received was so abundant that there are too many people to thank. My gratitude to Florence Debarre and Andrea Stephens (for their help preparing field materials), Lizzie Wolkovitch, Mark Vellend, Greg Crutsinger, Sean Naman, Thor Veen. Jenny Bryan's STAT545 course was a highlight, and taught me programming techniques which are used in all chapters of this thesis. I especially thank Dr. Rich Fitzjohn (my grad student mentor) for patiently teaching me R over many years, and for inventing many useful tools (including remake, which powers this thesis). Finally, I thank Matt Barbour, for many office conversations, transportation, and advice about ecology.

I have felt privileged to meet many interesting scientists outside our department, and many have had a positive impact on this dissertation. I'd especially like to thank the Software Carpentry, Rstudio and Ropensci communities, for creating useful tools (and healthy online spaces to discuss them). I thank my Twitter colleagues for endless light-hearted encouragement, especially Dr. Dave Harris who helped me create figure 3.1b. Thank you to all those who have encouraged me in the conviction that ecology must be an open and reproducible science.

Finally, thank you to my family. Thanks to Maureen and Arthur MacDonald, for loving and supporting me all my life. They gave me my love of science and nature -- and then quietly tolerated an endless parade of dead and dying animals and plants dragged into their house. My brother Brian and sister Marybeth were alongside me during the most difficult parts of the writing, and I can't thank you both enough. 

Most of all, thank you Angela MacDonald. She alone knows how challenging this dissertation was to finish. It would have been impossible without you. \emph{Tapadh leat, mo ghr\`adh}.


%%% Local Variables:
%%% TeX-master: "thesis.tex"
%%% TeX-PDF-mode: t
%%% End:
