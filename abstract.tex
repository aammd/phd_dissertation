\chapter*{Abstract}
\chaptermark{abstract}
\addcontentsline{toc}{chapter}{Abstract}

Many ecological communities show variation from place to place; understanding the causes of this variation is the goal of community ecology. Differences in community composition will be the result of both stochastic and deterministic processes. However, it is difficult to know to what degree, and under which circumstances, deterministic processes will shape community composition. In this thesis I combined observational and experimental approaches to quantify deterministic processes within a particular ecological community -- they phytotelmata of bromeliad plants. In my thesis I describe three studies at different scales of organization: 1) do organisms of different size respond equally to changes in their environment 2) how do predators interact to influence prey survival 3) what mechanisms underly the response of similar species to the same environmental gradient, bromeliad size. 

In Chapter 1, I tested an hypothesis developed from previous observational data - that smaller organisms respond less than larger ones to the same environmental gradient -- different bromeliad species that occur under different forest canopies. I placed identical communities of bacteria, zooplankton and insect communities into bromeliads in different habitats. I found that community composition diverged little for bacteria, more for zooplankton and most of all for macroinvertebrates. In my second chapter, I examined ecological determinism on a smaller scale -- within a single trophic level (macroinvertebrate predators). I found that predators may interfere with each other, reducing predation rates and increasing prey survival. In Chapter 3, I examine macroinvertebrate responses to bromeliad volume. I use a null model to demonstrate that species vary in their response to area more than can be explained by sampling effects alone. Then I discuss a detailed field experiment which showed that for at least one such pair, a difference in abiotic tolerances may be the plausible mechanism. 

Together these results illustrate when, and to what degree, bromeliad communities respond to deterministic factors. All three chapters first demonstrate a pattern, testing it against a suitable null distribution, before attempting to quantify possible mechanisms with a field experiment. This combination of observation and experiment is an approach which can contribute to our understanding of how ecological systems work. 
%%% Local Variables:
%%% TeX-master: "thesis"
%%% TeX-PDF-mode: t
%%% End:
