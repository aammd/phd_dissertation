\chapter{Conclusion}
\label{chap:conclusions}

\section{Overview of results}\label{overview-of-results}

What creates variation among ecological communities? In this thesis I
have attempted to first demonstrate a number of patterns using
observations, then showed that these patterns are non-trivial with an
appropriate null model, and finally tested those patterns with
controlled field experiments. In Chapter 2, I showed that organisms of
different size respond to different degrees to the same environmental
gradient. Bacterial communities changed very little in response to
habitat differences among bromeliads, while larger organism types
(zooplankton and insects) changed much more. In Chapter 3, I
demonstrated how predator phylogenetic diversity affects ecosystem
function and prey density in a bromeliad system. I showed that unrelated
predators show a nonadditive and negative effect on prey (detritivore)
survival. However, this change in prey density (i.e.~detritivore
density) did not generate an effect on ecosystem function. In Chapter 4,
I showed that related animals tend to be more different in their habitat
size distribution than expected by chance. I demonstrated that, in the
case of two similar chironomid larvae, this difference is caused by
their different responses to the environment: the two congenerics form a
generalist-specialist pair, with otherwise equivalent interactions with
each other and with predators.

Ecologists are confronted by a striking diversity of species
compositions, even within the same community type: not all species are
present everywhere, and even within local scales patches contain only
some of the possible species pool. Where does this variation come from?
It is created by a combination of processes, some of which are purely
numerical or stochastic, others which are deterministic. Vellend
\citep{Vellend2010b} suggests that all the many processes which
ecologists have considered as causes of this variation may be placed
into four categories: ecological selection, drift, speciation and
dispersal. My thesis is an attempt to detect the pattern of ecological
selection (i.e.~the most deterministic of these processes, defined
below) on a particular part of the life cycle of these organisms: the
part spent within the bromeliad mesocosm. However, the results I have
obtained and the patterns I have observed are also connected to the
other three processes. I will rely on Vellend's framework to organise
the remainder of this discussion.

\section{Ecological selection}\label{ecological-selection}

Ecological selection comprises those processes which favour the
occurrence of one species over others. This encompasses such processes
as ``habitat filtering'' (Chapter 2), ``niche partitioning'' (Chapter 4)
and species interactions (Chapter 3). Ecological selection is distinct
from natural selection in that it deals with the persistence of species
through time by demographic processes, not the persistence of genes
through time through inheritance. The environmental conditions of a
local community can determine at least some of the composition of the
species found there. In my thesis I used the naturally patchy mesocosms
found in bromeliads to examine how variation in community composition
arises. Bromeliads are spatially structured -- each individual bromeliad
is a discrete patch of habitat which varies in local environmental
characteristics and colonization histories, and this composite of
abiotic and biotic variables can combine to influence community
composition.

Abiotic variables determining species composition can vary at different
spatial scales. In Chapter 2 and 4 I measured the effect of the abiotic
environment at two scales -- among different bromeliad species in
different habitats (Chapter 2) and among different sizes of the same
bromeliad species (Chapter 4). Both chapters report that associated
environmental differences influence the survival of macroinvertebrate
larvae. Interestingly, this common role of environmental limitation was
found even though the two chapters differ in both the scale of the
gradient and the scale of the taxonomic diversity in the animals
considered. The experiment in Chapter 2 used a very ``steep''
environmental gradient -- different bromeliads in different parts of the
habitat -- and considered responses among all the macroinvertebrates
(across several orders). In contrast, Chapter 4 examined a relatively
subtle environmental gradient (bromeliads of different size, but the same species) and likewise contrasted
two species which were very similar in both taxonomy and morphology
(\emph{Polypedilum}). The study in Chapter 2 (regarding organism size and environmental filtering) presented a more coarse-grained view, reducing the variation
\emph{within} a habitat type to a single factor level (i.e.~bromeliad species/habitat type). Such a study might have
concluded that very similar macroinvertebrates (e.g. \emph{P. marcondesi} and
\emph{P. kaingang}) are able to coexist at that scale. However, Chapter 4
shows that within a single broad ``habitat type'' (i.e.~the same
bromeliad species in the same general habitat), the environmental
differences across different sizes are still important enough to
separate these two species. These two studies illustrate that the
effects of ecological determinism (ecological selection \emph{sensu}
Vellend) caused by the environment is very dependent on the spatial and
taxonomic scale being studied.

Biotic interactions can also impose ecological selection on the
composition of a local community. Negative interactions in particular
can create ecological selection, by limiting local composition to only
those species which can tolerate the interactions. Competition is
frequently considered an important negative interaction within trophic
levels. However, I recovered very little evidence for this process in my
field experiments. In Chapter 4, I did not detect any significant effect
of competition between two functionally similar \emph{Polypedilum}
larvae. There was more evidence for negative interactions within a
trophic level in chapter 3, where I showed that diverse predator
assemblages actually consume \emph{less} prey than monocultures. This is
not resource competition, but rather a kind of predator-predator
interference, possibly caused by the risk of intraguild predation.
Perhaps because of the small, confined nature of bromeliad communities,
such nonconsumptive effects are common between species, and can have
far-reaching consequences on both rates of predation and the functioning
of the entire ecosystem \citep{Atwood2014}. Even if competition is
important among bromeliad invertebrates, coexistence may not be
dependent on partitioning an environmental gradient - for example,
bromeliad size, or different bromeliad species. In fact, divergence
along gradients can sometimes result in competitive exclusion rather
than coexistence \citep{Mayfield2010}. If habitat partitioning of close
relatives is necessary for coexistence, abiotic tolerance traits must be
more labile than traits relating to competition. While this may occur in
allopatric speciation, the opposite pattern is expected in the case of
sympatric speciation. Although the evolutionary biogeography of
bromeliad macroinvertebrates is still unknown, there are examples of
speciation within bromeliads -- e.g.~among Dysticid beetles, whose
association with bromeliads extends back millions of years
\citep{Balke2008} -- and colonization of new species from completely
different habitats -- i.e.~mosquito species switched to bromeliads from
small container habitats \citep{Kitching2001}.

By far the most important negative interaction was predation. Predators
are an important part of the bromeliad community in most parts of the
world, and particularly on Ilha do Cardoso (the field site for Chapters
3 and 4) where they are more numerous, and more diverse, than anywhere else
where formal bromeliad surveys have been conducted (Bromeliad Working
Group, unpub. data). Despite having strong impacts on prey survival,
these predators imposed only weak ecological selection on the
invertebrate community -- consuming prey, but stochastically (i.e.~not
creating variation in species composition: Chapter 4). However,
predators did interact with each other in a negative way, resulting in
less overall predation when predator diversity was high(Chapter 3). Note
that there was no variation in predation effects in Chapter 2, since all
bromeliads contained homogeneous communities (each containing at least
one predator). However, because predators may respond differently than
their prey to the same environmental gradient -- e.g., via changes in
their metabolic rates at higher temperatures -- some of the
environmental effect we observed might have been in part the effect of
predation. This would be analogous to the previously-shown indirect
effects of drought on bromeliad insects via loss of predators
\citep{Amundrud2015}. These results (from all three chapters) suggest that
predators are able to have profound effects on bromeliad communities
once animals have colonized.

\subsection{Taxonomy}\label{taxonomy}

Biotic and abiotic variables only create deterministic selection on
species assemblages when species vary in traits that determine their
response. However, trait data are often lacking, as measurements of
species and individuals are rare. Often, ecologists use patterns of
phylogenetic diversity as a proxy for traits. In this framework, closely
related species are assumed to have similar resource acquisition traits,
and therefore are likely to competitively exclude each other. However,
close relatives might also be very similar in their environmental
tolerance traits and therefore may be likely to co-occur as they respond
in the same way to the same environmental gradient. The bromeliad fauna
of Cardoso lacks a good phylogeny, and therefore I have used taxonomic
categories as a loose guide. This requires the assumption that our
taxonomic categories are good representations of phylogenetic
relationships. Such an assumption may be problematic. The invertebrate
taxa found in bromeliads differ in divergence times by more than 200
million years, so branch lengths within a particular taxonomic level may
be very different. In Chapter 3, we tried to deal with this by using
published estimates for the dates of important nodes in the phylogeny of
predaceous taxa. Incomplete taxonomy poses a second problem: some of our
taxa are identified only to morphospecies, meaning that there may be
more congeneric pairs in this community than we considered. This would
bias our results if we have identified only the most morphologically
distinct subset of congenerics, missing cryptic congenerics which have
been argued to interact more neutrally \citep{Siepielski2010}. On the
other hand, if our selection of congenerics was unbiased, then future
identification of congenerics would only strengthen the power of our
analysis in Chapter 4. More generally, phylogenetic community ecology
assumes that species exclude each other through competition
\citep{Narwani2015}. However, competitive exclusion may be more rare than
ecologists assume, either because inter- and intra-specific competition
are similar \citep{Hubbell1997}, or because strong top-down effects
preclude resource limitation of intermediate trophic levels
\citep{Holt2004}. Therefore, we caution against inferring competition as
the driver of the pattern without evidence of prior competition
\citep{CahillJr.2008}.

\section{Other ecological processes}\label{other-processes}

In all chapters of this thesis, I have tried to measure deterministic
processes. However, there are three other processes which can create
variation in community composition: dispersal, speciation, and drift.
While none of these processes are directly measured in my thesis, they
are all controlled for, or otherwise inform subsequent analysis. More
importantly, as the other three processes create variation in natural
communities, they are critical future areas of study. Below, I briefly
summarize how my methods control for these effects, and also how my
results suggest hypotheses for how they may act in this system.

\subsection{Dispersal}\label{dispersal}

While the abiotic and biotic environments can determine survival of
organisms once they are found in bromeliads, the composition of these
communities is also determined by which animals arrive in the first
place. Dispersal can be either active or passive. In bromeliads, passive
dispersal is usually the action of vectors, such as birds or frogs,
which move from plant to plant. Active dispersal usually takes the form
of a female insect laying eggs in a nonrandom way. Females may select
bromeliads based on their presumed habitat quality for their offspring.
I did not measure female oviposition behaviour directly, but rather the
conditions for larval survival. However, since oviposition behaviour may
be an adaptation to optimize larval performance, the results of these
experiments suggest future hypotheses about how female insects select
oviposition sites.

If female choices are adaptations to maximize larval survival, they may
avoid habitats where their larvae face predation or unsuitable
environments. For example, female insects of many species will avoid
ovipositing eggs into bromeliads with predators inside them
\citep{Hammill2015} or above them \citep{Romero2010}. These
non-consumptive effects can be equal in magnitude to the effects of
predation itself. In a series of feeding trials (Chapter 3), I found
that damselflies consumed more individuals and more species than other
predator taxa. This strong effect of damselfly predation corroborates
previous results from multiple sites \citep{Petermann2015a} and helps explain why
adults avoid ovipositing in bromeliads with damselflies. However, the
results of Chapter 3 suggest a new question. Chapter 3 showed that
damselflies lower their feeding rate when in the presence of other
predators. If so, then this suggests that the identity of the
predator(s) in bromeliads is important for determining the expected
amount of predation, and therefore may influence oviposition decisions
by adults. For example, avoidance of damselflies might be lessened by
the presence of a leech or tabanid. Predator colonization may also be
directly affected: the female damselflies may themselves avoid other
predators. This raises the tantalizing prospect of a
behaviourally-mediated trophic cascade, where fear drives multiple
interactions between trophic levels. Understanding which colonizing
species are important, and the relative importance of colonization
vs.~within-bromeliad predation, will require more detailed field
experiments.

\subsection{Speciation}\label{speciation}

While evolutionary history (as approximated by phylogeny and taxonomy)
may correlate with existing trait variation that is important in
communities, where do these traits come from in the first place?
Speciation introduces new species, which might have different traits
than those already existing in communities. Some organisms found in
bromeliads have close relatives that live in other habitats
(e.g.~caddisflies), while other organisms belong to whole clades of
bromeliad specialists, which speciated after adapting to bromeliads
(e.g. \emph{Mecistogaster} spp. damselflies). Integrating local ecology
with patterns of speciation -- or, similarly, asking how historical
contingency shapes contemporary responses -- has been suggested as an
important future direction in community phylogenetics.
Gerhold et al. \citet{Gerhold2015} call these approaches
``phylogenetic-pattern-as-response'' vs
``phylogenetic-pattern-as-cause''. Understanding the historical
phylogeography of bromeliads would place the results from studies of
species interactions (such as Chapter 3 and 4) into context, allowing us
to relate, for example, interaction strength to the time of association
of two species. In order to fully exploit this advantage of bromeliads,
we would first need a more complete taxonomy and a better phylogeny of
the identified species.

\subsection{Drift}\label{drift}

Drift is the most difficult ecological process to measure, but might be
especially important in small, fragmented habitats like bromeliads.
Drift is caused by the random sequence of demographic events in the life
cycle of organisms, including death, reproduction and dispersal. Drift
therefore generates variation in bromeliad communities from the moment
propagules arrive in bromeliads. Even when dispersal is active, the
number of eggs a female oviposits in a bromeliad may vary among species
and bromeliads. Chironomids, for example, can produce between 80 and 200
eggs \citep{}. We have little information about demographic
rates after oviposition, although studies like that described in Chapter
4 can help us to measure growth and emergence rates. However such
demographic studies are difficult to do for the many species in a
multispecies community. One option for quantifying the role of drift in
multispecies communities is to instead compare compositional change in
identical communities over time \citep{Vellend2010b}. The importance of
drift may also vary across the range of bromeliaceae (across the
Neotropics). The null simulations I performed in this research
(e.g.~Chapter 4) highlight that the patterns we detect are the
result of two different statistical distributions: first, the distribution bromeliad sizes and second, the relative abundance distribution of
bromeliad-dwelling animals. Across habitats, bromeliad size distributions will vary depending on site characteristics and traits of the
bromeliad species present. Meanwhile, the shape of the relative
abundance distribution of insects also varies across sites. Since drift
is most important for rare species and in small habitat patches, the
shape of these two distributions may be very important in determining
the degree to which drift structures communities.

In summary, in this thesis I demonstrated that both the abiotic and the
biotic environment determine the community composition of invertebrates.
Across habitats, invertebrates are more sensitive to environmental
variation than zooplankton and bacteria. Within a habitat, bromeliads of
different sizes may have different environments, with animals showing
specialist-generalist responses to this habitat axis. Biotic
interactions, especially predation, act within and between trophic
levels to influence community composition and/or abundance. Each chapter
of this thesis has demonstrated a pattern in species distribution
against a null model, then quantified the interactions that might have
caused this pattern. Both approaches are key to describing and
explaining the diversity of community types we see in nature.
